\documentclass[a4paper,11pt]{article}

% --- Packages de base ---
\usepackage[utf8]{inputenc}
\usepackage[T1]{fontenc}
\usepackage[french]{babel}
\usepackage{lmodern}
\usepackage{microtype}
\usepackage{geometry}
\geometry{margin=2.5cm}
\usepackage{setspace}
\setstretch{1.08}
\usepackage{enumitem}
\setlist{itemsep=2pt,topsep=4pt}
\usepackage{hyperref}
\hypersetup{
  colorlinks=true,
  linkcolor=black,
  urlcolor=blue,
  pdftitle={Proposition DYS-FR (version client)},
  pdfauthor={Tarazona Javier, Chacón José}
}

% --- Mise en forme titres ---
\usepackage{titlesec}
\titleformat{\section}{\large\bfseries}{\thesection.}{0.5em}{}
\titleformat{\subsection}{\normalsize\bfseries}{\thesubsection}{0.5em}{}

% --- Métadonnées ---
\title{\textbf{Proposition de projet — DYS-FR (version client)}}
\usepackage{authblk}

\title{\textbf{Proposition de projet — DYS-FR (version client)}}
\author{Tarazona Javier}
\author{Chacón José}
\affil{\small Étudiants de 2\textsuperscript{e} année, ENSTA Paris}
\date{\small \today}

\begin{document}
\maketitle
\vspace{-0.75em}
\noindent\rule{\linewidth}{0.4pt}

\section{Résumé exécutif}
\textbf{DYS-FR} est une application \textbf{100\,\% hors-ligne} qui clarifie et corrige des textes en français \textbf{sans ajouter d’informations}. Elle s’installe sur Windows en un clic et fonctionne sur des ordinateurs portables standards. L’utilisateur colle son texte, clique sur \textbf{«\,Améliorer\,»}, puis visualise les \textbf{modifications} et le \textbf{texte final}.

\section{À qui s’adresse l’outil ?}
\begin{itemize}
  \item Étudiants, enseignants, rédacteurs, équipes administratives.
  \item Public DYS et toute personne souhaitant un \textbf{retour discret, local et fiable} sur ses écrits.
\end{itemize}

\section{Valeur ajoutée}
\begin{itemize}
  \item \textbf{Confidentialité totale} : aucune donnée ne sort de l’ordinateur.
  \item \textbf{Simplicité} : interface claire, un bouton, résultat immédiat.
  \item \textbf{Fiabilité} : corrections prudentes, \textbf{zéro invention} (noms, dates, chiffres préservés).
  \item \textbf{Accessibilité} : taille de police, interligne, \textbf{contraste élevé}.
\end{itemize}

\section{Fonctionnement (vue utilisateur)}
\begin{enumerate}
  \item \textbf{Coller} un texte en français.
  \item Choisir le mode :
  \begin{itemize}
    \item \textbf{Léger (rapide)} : règles sûres, très fluide.
    \item \textbf{Intelligent (hybride)} : plus de finesse, toujours sans ajout d’informations.
  \end{itemize}
  \item Cliquer \textbf{«\,Améliorer\,»}.
  \item Consulter :
  \begin{itemize}
    \item \textbf{Diff} (changements en rouge/vert),
    \item \textbf{Texte final} (prêt à copier).
  \end{itemize}
\end{enumerate}

\section{Mode Intelligent (hybride) — nos 2 modèles IA}
En mode \textbf{Intelligent}, l’application s’appuie sur \textbf{deux modèles locaux de traitement du langage} (NLP), exécutés \textbf{hors-ligne} sur la machine de l’utilisateur :
\begin{enumerate}
  \item \textbf{Mistral-7B-Instruct v0.3 (quantifié Q4)}\\
  \emph{Rôle} : modèle principal pour une \textbf{qualité linguistique élevée}.\\
  \emph{Avantage} : excellentes reformulations \textbf{sans trahir} le contenu.\\
  \emph{Utilisation} : choisi par défaut quand la machine dispose de ressources suffisantes.
  \item \textbf{Gemma 3 4B}\\
  \emph{Rôle} : \textbf{alternatif plus léger} pour des ordinateurs modestes.\\
  \emph{Avantage} : \textbf{démarrage plus rapide} et empreinte mémoire réduite, tout en respectant strictement la contrainte \textbf{«\,pas d’ajout d’information\,»}.\\
  \emph{Utilisation} : \textbf{bascule automatique} si Mistral n’est pas disponible ou si les ressources sont limitées.
\end{enumerate}

\noindent\textbf{Sélection \& sécurité.}
L’application \textbf{détecte} les capacités de la machine et choisit le modèle le plus adapté. En cas de difficulté technique (manque de mémoire, indisponibilité), elle \textbf{passe automatiquement} de Mistral-7B vers Gemma 3 4B, sans intervention de l’utilisateur. Les deux modèles fonctionnent \textbf{en local} : \textbf{aucun envoi de texte} vers Internet. Après réécriture, un \textbf{contrôle automatique} vérifie que les erreurs n’augmentent pas et que \textbf{noms/dates/chiffres} restent inchangés ; sinon, l’outil \textbf{revient} à une version plus sûre.

\section{Confidentialité \& sécurité}
\begin{itemize}
  \item \textbf{Hors-ligne par conception} (zéro cloud).
  \item Pas de compte, pas d’inscription.
  \item Noms propres, chiffres et dates \textbf{conservés}.
\end{itemize}

\section{Expérience \& accessibilité}
\begin{itemize}
  \item Zone de texte large, bouton \textbf{«\,Améliorer\,»}.
  \item \textbf{Diff visuel} (avant/après).
  \item Réglages : \textbf{taille de police}, interligne, contraste.
\end{itemize}

\section{Livrables (MVP)}
\begin{itemize}
  \item \textbf{Installateur Windows (.exe)} avec tout inclus.
  \item Application prête à l’emploi (modes \textbf{Léger} et \textbf{Intelligent}).
  \item \textbf{Guide d’utilisation} (PDF court, 3–5 captures).
\end{itemize}

\section{Périmètre (MVP)}
\textbf{Inclus}
\begin{itemize}
  \item Correction/clarification \textbf{sans ajout d’informations}.
  \item Fonctionnement \textbf{100\,\% local}.
  \item Interface simple (\textbf{Diff} + \textbf{Texte final}).
\end{itemize}

\section{Plan de travail (gain de temps)}
\textbf{Phase 0 — Setup (3--5 h)}
\begin{itemize}
  \item Repo, venv, \texttt{requirements.txt}, structure dossiers, \texttt{config.yaml}.
\end{itemize}

\textbf{Phase 1 — Ressources hors-ligne (5--7 h)}
\begin{itemize}
  \item Copier JAR de LT, \textbf{JRE} embarqué, spaCy FR.
  \item Scripts de vérification des chemins.
\end{itemize}

\textbf{Phase 2 — Noyau NLP (12--16 h)}
\begin{itemize}
  \item Connecteur LanguageTool (client) + \textbf{filtre} de suggestions.
  \item \textbf{Composant LLM local} (Mistral/Gemma) + prompts contraints («\,ne pas modifier marqueurs/format\,») + \textbf{fallback}.
  \item Normalisation + post-traitement FR.
\end{itemize}

\textbf{Phase 3 — Garde-fous (8--12 h)}
\begin{itemize}
  \item \texttt{spaCy} NER/PROPN.
  \item Regex nombres/dates.
  \item Ratio de tokens nouveaux (lemmes).
  \item Fallback sûr.
\end{itemize}

\textbf{Phase 4 — UI Streamlit (6--8 h)}
\begin{itemize}
  \item Écran unique : entrée $\rightarrow$ bouton $\rightarrow$ Diff / Résultat.
  \item Contrôles d’accessibilité de base.
  \item Diff avec \texttt{difflib} / \texttt{diff-match-patch}.
\end{itemize}

\textbf{Phase 5 — Runner + Packaging (10--14 h)}
\begin{itemize}
  \item \texttt{runner.py} (LT + \textbf{serveur LLM} + ouverture navigateur + fermeture) avec \textbf{détection matériel} et \textbf{offload GPU} quand présent.
  \item PyInstaller (exe) + Inno Setup (installateur) avec ressources \textbf{JRE/LT/spaCy/LLM} et \textbf{cache de modèles}.
  \item Test d’installation sur Windows «\,propre\,».
\end{itemize}

\textbf{Phase 6 — QA \& Docs (6--8 h)}
\begin{itemize}
  \item Tests e2e/adversariaux.
  \item README/Guide d’utilisation (3--5 captures).
  \item Notes de confidentialité (hors-ligne).
\end{itemize}

\textbf{Marge (10--15\,\%) (6--9 h)}\\[2pt]
\noindent\textbf{Total estimé (Windows uniquement)} : \textbf{50--71 heures} (typique \textbf{60--65 h}).

\section{Critères d’acceptation (exemples)}
\begin{itemize}
  \item Installation et ouverture en \textbf{1 clic}.
  \item Texte de 1--2 phrases amélioré en \textbf{$<$ 1 s} en mode \textbf{Léger}.
  \item \textbf{Aucune invention} : chiffres/dates/noms identiques à l’original.
  \item Changements visibles dans l’onglet \textbf{Diff}.
  \item Fonctionnement \textbf{hors-ligne} (aucun appel réseau).
\end{itemize}

\section{Prérequis côté client}
\begin{itemize}
  \item PC Windows récent (laptops acceptés).
  \item Autoriser l’installation d’un exécutable signé (standard entreprise).
\end{itemize}

\vfill
\noindent\rule{\linewidth}{0.4pt}\\
\small \textit{Cette proposition est préparée et sera développée par \textbf{Tarazona Javier} et \textbf{Chacón José}, étudiants de 2\textsuperscript{e} année à \textbf{ENSTA Paris}.}
\end{document}